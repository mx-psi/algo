\section{Contenedores en un barco}

\begin{frame}{Problema}
Rellenar un buque con carga limitada ($K$) con un conjunto
de contenedores $c_1,\dots, c_n$ con pesos $p_1, \dots, p_n$.

\begin{description}
 \item[Entrada:] Vector de pesos de los contenedores, $p$ y capacidad total $K$
 \item[Salida:] Vector con los contenedores elegidos
\end{description}
\end{frame}

\begin{frame}[fragile]{Estructura de datos}
\lstinputlisting[firstline=11, lastline=19]{cpps/contenedores.cpp}
\end{frame}

\subsection{Maximizar el número de contenedores cargados}

\begin{frame}{Solución}
\begin{itemize}
  \item Basta coger los contenedores de menor peso hasta rellenar el buque
  \item Emparejamos cada elemento con su posición y ordenamos en función de los pesos.
  \item \textbf{Eficiencia:} $O(n\log(n))$
\end{itemize}
\end{frame}

\begin{frame}[fragile]{Código}
\lstinputlisting[firstline=38, lastline=51]{cpps/contenedores.cpp}
\note{\texttt{menor} compara dos contenedores según su peso, indicando cuál es menor}
\end{frame}

\begin{frame}{Optimalidad de la solución}
Este criterio siempre halla la solución con un mayor número de contenedores.
%%TODO: ¿Lo ponemos? ¿Cómo?
\end{frame}

\subsection{Maximizar el número de toneladas cargadas}

\begin{frame}{Estrategia}
\begin{itemize}
  \item Cogemos el contenedor de mayor peso que quepa en el buque
  \item El algoritmo es muy similar al apartado anterior.
\end{itemize}
\end{frame}

\begin{frame}[fragile]{Código}
\lstinputlisting[firstline=53, lastline=67]{cpps/contenedores.cpp}
\end{frame}

%% TODO: Justificación chusquera de por qué hemos elegido esta heurística

%% TODO: Explicar el algoritmo de fuerza bruta
