\section{El problema del viajante de comercio}

El problema del viajante de comercio consiste en hallar el recorrido con distancia
mínima en un conjunto de ciudades que pase por todas las ciudades y regrese al punto
inicial. \\

La implementación de los algoritmos es de la forma:
\begin{description}
 \item[Entrada:] Ficheros con ciudades indicadas como puntos en el plano según sus
 coordenadas.
 \item[Salida:] \texttt{vector<int>} con el orden en el que se recorren las ciudades.
\end{description}

Para todos los algoritmos utilizaremos una estructura de datos que nos permita manejar
el problema: la clase \texttt{Grafo} (en el fichero \texttt{grafo.h}). Un grafo consta
de:

\begin{itemize}
  \item Una \textbf{cantidad de nodos}, almacenada en el atributo \texttt{nodos}.
  \item Una \textbf{matriz de pesos}, almacenada en el vector \texttt{lados}.
\end{itemize}

La interfaz nos permite acceder y modificar estos datos con mayor facilidad.
Mediante los métodos \texttt{setPeso} y \texttt{peso} accederemos al peso de un
lado del grafo, y el método \texttt{pesosDesdeCoordenadas} nos permite inicializar el
grafo utilizando el formato de datos en el que aparece el problema: calculando
la distancia entre cualesquiera dos ciudades y añadir esta como peso de ese lado:

\lstinputlisting[firstline=47, lastline=51]{cpps/grafo.h}

Adicionalmente, la función \texttt{longitud} calcula la longitud de un camino dado.


\subsection{Vecino más cercano}

Utilizando la estructura de datos explicada anteriormente, la resolución del problema
utilizando la heurística del vecino más cercano quedará:

\lstinputlisting[firstline=16, lastline=39]{cpps/tsp.cpp}

Tomamos como ciudad inicial el nodo 0 y almacenamos en la lista \texttt{disponibles}
las ciudades no visitadas. A continuación, mientras queden ciudades disponibles,
recorremos la lista buscando aquella ciudad con distancia mínima a la última.

\subsection{Estrategias de inserción}

\subsection{Colonia de hormigas}

Como solución adicional propuesta por el equipo utilizamos un algoritmo basado en
colonias de hormigas. Este algoritmo se inspira en la comunicación por feromonas
de una colonia de hormigas para encontrar el camino mínimo hacia una fuente de comida. \\

Implementamos la Colonia (en el fichero \texttt{colonia.h}). Una Colonia consta de:

\begin{itemize}
  \item Un grafo de \textbf{distancias} entre las ciudades.
  \item Un grafo con las \textbf{feromonas} de cada camino, inicialmente arbitrarias.
  \item Una serie de constantes $\alpha, \beta, \texttt{EVAPORACION}, C, P$.
\end{itemize}

Las constantes controlan el comportamiento del algoritmo:

\begin{description}
  \item[$\alpha$] es el peso que tienen las feromonas.
  \item[$\beta$] es el peso que tienen la distancias.
  \item[\texttt{EVAPORACION}] es el coeficiente de evaporación de las feromonas.
  \item[$C$] determina cuánta feromona se añade a un camino.
  \item[$P$] es un flotante entre $0$ y $1$. Durante el progreso de un recorrido, hay
  una probabilidad de $1-P$ de que la hormiga opte por tomar directamente el camino
  más corto a otro nodo.
\end{description}
