\section{El problema del viajante de comercio}

\begin{frame}{Problema}
Hallar el recorrido con distancia mínima en un conjunto de
ciudades que pase por todas las ciudades y regrese al punto inicial.
\end{frame}

\begin{frame}{Algoritmos}
\begin{description}
 \item[Entrada:] Ficheros con ciudades indicadas como puntos en el plano según sus
 coordenadas.
 \item[Salida:] \texttt{vector<int>} con el orden en el que se recorren las ciudades.
\end{description}
\end{frame}

\begin{frame}{Grafos}

Un grafo consta de:

\begin{itemize}
  \item Una \textbf{cantidad de nodos}, almacenada en el atributo \texttt{nodos}.
  \item Una \textbf{matriz de pesos}, almacenada en el vector \texttt{lados}.
\end{itemize}
\end{frame}

\begin{frame}[fragile]{Código}
\lstinputlisting[firstline=47, lastline=51]{cpps/grafo.h}
\end{frame}

\subsection{Algoritmos}

\begin{frame}[fragile]{Vecino más cercano}
\lstinputlisting[firstline=16, lastline=39]{cpps/tsp.cpp}
\note{Tomamos como ciudad inicial el nodo 0 y almacenamos en la lista \texttt{disponibles}
las ciudades no visitadas. A continuación, mientras queden ciudades disponibles,
recorremos la lista buscando aquella ciudad con distancia mínima a la última, la
cual añadimos al trayecto y eliminamos de la lista de disponibles.}
\end{frame}

\begin{frame}{Estrategias de inserción}
%% TODO: Explicación del resto de la estrategia
\end{frame}

\begin{frame}{Colonia de hormigas}
Este algoritmo se inspira en la comunicación por feromonas
de una colonia de hormigas para encontrar el camino mínimo hacia una fuente de comida.
%% TODO: Explicación de cosas
\end{frame}

\begin{frame}{Colonia}
Implementamos la Colonia (en el fichero \texttt{colonia.h}). Una Colonia consta de:

\begin{itemize}
  \item Un grafo de \textbf{distancias} entre las ciudades.
  \item Un grafo con las \textbf{feromonas} de cada camino, inicialmente arbitrarias.
  \item Una serie de constantes $\alpha, \beta, \texttt{EVAPORACION}, C, P$.
\end{itemize}
\end{frame}

\begin{frame}{Constantes}
\begin{description}
  \item[$\alpha$] es el peso que tienen las feromonas.
  \item[$\beta$] es el peso que tienen la distancias.
  \item[\texttt{EVAPORACION}] es el coeficiente de evaporación de las feromonas.
  \item[$C$] determina cuánta feromona se añade a un camino.
  \item[$P$] es un flotante entre $0$ y $1$. Durante el progreso de un recorrido, hay
  una probabilidad de $1-P$ de que la hormiga opte por tomar directamente el camino
  más corto a otro nodo.
\end{description}
\end{frame}

\begin{frame}{Comparativa de los algoritmos}
  %%TODO: Cosas
\end{frame}
