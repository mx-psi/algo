
\section{Contenedores en un barco}

El problema consiste en rellenar un buque con carga limitada ($K$) con un conjunto
de contenedores $c_1,\dots, c_n$ con pesos $p_1, \dots, p_n$ con un cierto objetivo.
\\

La implementación de los algoritmos es de la forma:
\begin{description}
 \item[Entrada:] Vector de pesos de los contenedores, $p$ y capacidad total $K$
 \item[Salida:] Vector con los contenedores elegidos
\end{description}

Utilizaremos también una estructura de datos simple llamada \texttt{Cont} que almacena un
contenedor:

\lstinputlisting[firstline=11, lastline=19]{cpps/contenedores.cpp}

\subsection{Maximizar el número de contenedores cargados}

Para este caso nos basta coger siempre el contenedor de menor peso disponible de entre los
que no hemos elegido ya.
Para hacer que el algoritmo sea eficiente emparejamos cada elemento
del vector de entrada con su posición (el contenedor que representa)
y ordenamos este vector en función de los pesos:

\lstinputlisting[firstline=38, lastline=51]{cpps/contenedores.cpp}

La función \texttt{menor} nos permite comparar dos contenedores según su peso,
indicando cuál es menor.

La eficiencia del algoritmo es $O(n\log(n))$, ya que la operación de ordenado es
$O(n\log(n))$ y el bucle posterior es de eficiencia lineal.

\newpage

\subsubsection{Optimalidad de la solución}

Este criterio (coger el de menor peso hasta que no quepan más) \textbf{siempre obtiene la
mejor solución} (la solución con un mayor número de contenedores).
Llamemos a nuestra solución $o_1, \dots, o_k$ y sea $s_1, \dots, s_m$ otra solución
cualquiera. Supongamos que $m > k$. Podemos asumir sin pérdida de generalidad que las
soluciones están ordenadas por pesos de menor a mayor.

Como los pesos de nuestra solución son mínimos, es claro que $\forall i < k: o_k \leq s_k$.
Por tanto también tendríamos como posible la solución $o_1, \dots o_k, s_{k+1}, \dots s_m$.
Sea $o_{k+1}$ el contenedor que no está en nuestra solución que tenga el menor peso.
Es claro que $o_{k+1} \leq \sum_{i > k} s_i$, por lo que tendríamos que
$o_1, \dots, o_{k+1}$ es solución.

Esto es una contradicción ya que por construcción nuestra solución toma los contenedores
de peso mínimo hasta que añadir uno más suponga sobrepasar la capacidad. Por tanto nuestra
solución es la solución con el mayor número de contenedores posibles.

\subsection{Maximizar el número de toneladas cargadas}

En este caso empleamos como estrategia \textit{greedy} coger los contenedores en orden
decreciente de peso: empezamos con el contenedor más pesado que quepa en el buque
y continuamos ordenadamente añadiendo el contenedor más pesado de entre los que quepan.

El código es muy similar a la solución anterior, salvo que cambiamos la condición
que ordena la lista inicialmente para ordenarla de forma decreciente en función del peso
y porque necesitamos realizar, en cada paso, la comprobación de que el contenedor que
queremos añadir no exceda la capacidad restante del buque:

\lstinputlisting[firstline=53, lastline=67]{cpps/contenedores.cpp}

%% TODO: Justificación chusquera de por qué hemos elegido esta heurística

%% TODO: Explicar el algoritmo de fuerza bruta
