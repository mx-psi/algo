
\section{Contenedores en un barco}

El problema consiste en rellenar un buque con carga limitada ($K$) con un conjunto
de contenedores $c_1,\dots, c_n$ con pesos $p_1, \dots, p_n$ con un cierto objetivo

La implementación de los algoritmos es de la forma:
\begin{description}
 \item[Entrada:] Vector de pesos de los contenedores, $p$ y capacidad total $K$
 \item[Salida:] Vector con los contenedores elegidos
\end{description}

\subsection{Maximizar el número de contenedores cargados}

Para este caso nos basta coger siempre el contenedor de menor peso disponible de entre los
que no hemos elegido ya.
Para hacer que el algoritmo sea eficiente emparejamos cada elemento
del vector de entrada con su posición (el contenedor que representa)
y ordenamos este vector en función de los pesos:

%\lstinputlisting[firstline=, lastline=]{cpps/contenedores.cpp}

\subsection{Maximizar el número de toneladas cargadas}
