\section{Estación de ITV}

En este problema se presenta una estación de ITV de $m$ líneas, entre las que
se pretende repartir un grupo de $n$ coches tales que cada coche requiere un
tiempo de inspección $T_i$ (independiente de la línea a la que se haya asignado)
de forma que el tiempo transcurrido desde el inicio hasta que terminen las
inspecciones sea mínimo. Por ello, se buscará organizar los coches en las $m$
líneas buscando que el máximo de la suma de los tiempos de inspección de cada
línea sea mínimo. \\

La implementación de los algoritmos es de la forma:
\begin{description}
\item[Entrada:] Vector de tiempos de los coches $T$ y número de líneas $m$
\item[Salida:] Vector que indica en la posición $i$ a qué línea va el coche $i$-ésimo
\end{description}

Proponemos un algoritmo que vaya asignando cada vehículo a cada línea, guarde la solución con el mínimo tiempo que se haya encontrado y vuelva atrás cuando la solución actual no pueda ser mejor que la de menor tiempo encontrada hasta el momento. \\

Para obtener una cota inicial utilizaremos un algoritmo greedy, que se limita a insertar cada coche, de mayor a menor tiempo, en la línea que menos ocupada se encuentre. El orden escogido busca que sea más fácil nivelar las líneas:

\lstinputlisting[firstline=38, lastline=67]{cpps/itv.cpp} % TODO: comprobar que no se mueva el código

% TODO: describir el procedimiento

% TODO: estudio empírico de la eficiencia

% TODO: comparar con el greedy
% TODO: ¿comparar con fuerza bruta mal enfocada?