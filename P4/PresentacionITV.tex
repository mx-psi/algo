\section{Estación de ITV}

\begin{frame}{Problema}
\begin{itemize}
	\item Una estación de ITV tiene $m$ líneas.
	\pause
	\item Hay $n$ vehículos que quieren repartirse entre las líneas. El vehículo $i$
	requiere un tiempo $T_i$ para ser inspeccionado.
	\pause
	\item Se pretende repartir los vehículos en las líneas de forma que la última
	inspección termine lo antes posible.
\end{itemize}
\pause

Presentaremos algoritmos con el formato:
\begin{description}
 \item[Entrada:] Vector de tiempos de los coches $T$ y número de líneas $m$
 \item[Salida:] Vector que indica en la posición $i$ a qué línea va el coche $i$-ésimo
\end{description}
\end{frame}

\subsection{Algoritmos}

\begin{frame}{Greedy}
	El greedy se usará dentro del algoritmo backtrack, además de servir como referencia para comparación de ambos.
	
	\pause
	
	Funcionamiento:
	\begin{itemize}
		\item Se ordenan los coches de mayor a menor tiempo.
		\item Se introduce el coche de mayor tiempo en la línea que se encuentre menos ocupada.
	\end{itemize}
\end{frame}

\begin{frame}{Backtrack}
	Se mantienen dos vectores de enteros que indica en la posición $i$ a qué línea va el coche $i$: uno con la mejor solución encontrada hasta el momento y otro cambiante, al que se le añaden y eliminan elementos durante la ejecución.
	
	También mantiene la cantidad de tiempo que se ha ocupado en cada línea durante el desarrollo.
	
	\pause
	
	En cada momento, intenta meter el primer vehículo no asignado en cada una de las líneas, y si, una vez introducido en una línea, esa línea tiene un tiempo de ocupación menor que el máximo tiempo de ocupación de la mejor solución, se sigue explorando esa posibilidad. Si no, se descarta.
\end{frame}

\begin{frame}{Backtrack}
	El programa solo comprueba la longitud de la línea actual. Es posible que alguna de las otras líneas tenga una carga no menor que la de la mejor solución encontrada hasta el momento. Pero comprobar todas las líneas dispara el tiempo de ejecución y no compensa.
	
	\pause
	
	Como primera mejor solución se utiliza el resultado greedy.
\end{frame}

\subsection{Eficiencia}

\begin{frame}{Eficiencia}
	\begin{itemize}
		\item Problema: \textbf{Tamaño del árbol}
		\pause
		\item El tiempo crece al aumentar las \textbf{líneas} y/o los \textbf{coches}
		\pause
		\item ¿Solución para estudiar la eficiencia?
		\pause
		\item \textbf{Fijamos} el número de filas y dejamos variable el número de coches.
		
		En este caso hemos fijado el número de líneas a 10:
	\end{itemize}
\end{frame}

\begin{frame}{Tiempo de ejecución}
	\begin{center}
		\includegraphics[width = \linewidth]{img/itvEficiencia}
	\end{center}
\end{frame}

\begin{frame}{Comparación con Greedy}
	\begin{itemize}
		\item Vamos a analizar qué hemos ganado y qué hemos perdido.
		\pause
		\item Está claro que los resultados son mejores, pero ¿y el tiempo?
		\pause
		\item Para $5$ líneas y $18$ coches el greedy tarda varios órdenes de magnitud menos.
		\item \textbf{¿Cuánto hemos ganado exactamente?}
		

	\end{itemize}
	\begin{center}
			\begin{tabular}{c|c|c}
				Coches	& Greedy  & Backtracking  \\ 
				16  & 182.83 & 175.91 \\
				17	& 190.33 & 183.56 \\ 
				18	& 209.09 & 200.59
			\end{tabular}
			
	\end{center}

\end{frame}

\begin{frame}{Gráfica Greedy vs Backtracking}
	\includegraphics[width = \linewidth]{./img/comp.pdf}
	
	El número de líneas se ha fijado a $5$.
\end{frame}

