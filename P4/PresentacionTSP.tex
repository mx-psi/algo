\section{El problema del viajante de comercio}

\begin{frame}{Problema}
Hallar el recorrido con distancia mínima en un conjunto de
ciudades que pase por todas las ciudades y regrese al punto inicial.

\includegraphics[width=.5\textwidth]{img/Francia} \centering
\end{frame}

\begin{frame}{Algoritmos}
\begin{description}
 \item[Entrada:] Ficheros con ciudades indicadas como puntos en el plano según sus
 coordenadas.
 \item[Salida:] \texttt{vector<int>} con el orden en el que se recorren las ciudades.
\end{description}
\end{frame}

% TODO: todo lo posterior

\begin{frame}{\texttt{Grafo}}
Un grafo consta de:

\begin{itemize}
  \item Una \textbf{cantidad de nodos}: \texttt{nodos}.
  \item Una \textbf{matriz de pesos}: \texttt{lados}.
\end{itemize}
\end{frame}

\subsection{Ramificación y acotación}

\begin{frame}{Algoritmo general}
  Iniciamos:
  \begin{itemize}
    \item La cola con la solución parcial \texttt{[0]}
    \item La mejor solución con un algoritmo \textit{greedy} (vecino más cercano)
  \end{itemize}
\end{frame}

\begin{frame}{Algoritmo general}
  Mientras la cola no esté vacía, coge el primer elemento:
  \begin{itemize}
    \item Si \textbf{puede formarse una solución completa}, comprueba si es mejor que la mejor encontrada. En tal caso actualizala y borra los nodos con peor cota.
    \item En \textbf{otro caso}, para cada ciudad no visitada forma una nueva solución parcial. Añádela a la cola si su cota es mejor que la mejor solución.
  \end{itemize}
\end{frame}

\subsubsection{Cotas}

\begin{frame}{Cota del mínimo}
  Calcular la longitud del recorrido inicial y suma la menor distancia de cada ciudad no incluida en la solución parcial con sus relacionadas.
\end{frame}

\begin{frame}{Arbol generador}
  Calcular la longitud del recorrido que se lleva hasta ese momento, y suma la longitud del árbol generador minimal de los nodos que faltan junto al último nodo del camino. Además, se suma la mínima longitud del primer nodo a otro más.

  %% Aquí pintad en pizarra que si no, no lo entiende ni dios, y es una tontería
\end{frame}

\subsection{Vuelta atrás}

\begin{frame}{Backtracking}
  Desde un nodo inicial, vamos tomando todas las posibilidades mediante una llamada recursiva. Si al añadir un nodo hace que el camino mida más que nuestra mejor longitud hasta el momento, se deja esa posibilidad, y se vuelve atrás.
\end{frame}

\begin{frame}[fragile]{Llamada inicial backtracking}
\begin{lstlisting}
vector<int> tsp_backtracking(const Grafo<peso_t>& g) {
	vector<int> primera_solucion = tsp_greedy(g);
	int mejor_longitud = longitud(primera_solucion,g);
	vector<int> inicial = {0};
	vector<int> solucion; // Se modifica

	tsp_back_rec(solucion,inicial,mejor_longitud,g);

	return solucion;
}
\end{lstlisting}
\end{frame}

\subsection{Comparativa de los algoritmos}

\begin{frame}{5 ciudades (tiempos)}
\includegraphics[width=\textwidth]{img/barras_e-h-c-a-c5_t}
\end{frame}

\begin{frame}{5 ciudades (nodos expandidos)}
\includegraphics[width=\textwidth]{img/barras_e-h-c-a-c5_nodos}
\end{frame}

\begin{frame}{5 ciudades (Podas)}
\includegraphics[width=\textwidth]{img/barras_e-h-c-a-c5_poda}
\end{frame}

\begin{frame}{5 ciudades (Cola)}
\includegraphics[width=\textwidth]{img/barras_e-h-c-a-c5_cola}
\end{frame}

\begin{frame}{10 ciudades (tiempos)}
\includegraphics[width=\textwidth]{img/barras_ulysses10_t}
\end{frame}

\begin{frame}{10 ciudades (nodos expandidos)}
\includegraphics[width=\textwidth]{img/barras_ulysses10_nodos}
\end{frame}

\begin{frame}{10 ciudades (Podas)}
\includegraphics[width=\textwidth]{img/barras_ulysses10_poda}
\end{frame}

\begin{frame}{10 ciudades (Cola)}
\includegraphics[width=\textwidth]{img/barras_ulysses10_cola}
\end{frame}

\begin{frame}{12 ciudades (tiempos)}
\includegraphics[width=\textwidth]{img/barras_branchambao12_t}
\end{frame}

\begin{frame}{12 ciudades (nodos expandidos)}
\includegraphics[width=\textwidth]{img/barras_branchambao12_nodos}
\end{frame}

\begin{frame}{12 ciudades (Podas)}
\includegraphics[width=\textwidth]{img/barras_branchambao12_poda}
\end{frame}

\begin{frame}{12 ciudades (Cola)}
\includegraphics[width=\textwidth]{img/barras_branchambao12_cola}
\end{frame}
