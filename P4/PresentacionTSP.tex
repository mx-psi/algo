\section{El problema del viajante de comercio}

\begin{frame}{Problema}
Hallar el recorrido con distancia mínima en un conjunto de
ciudades que pase por todas las ciudades y regrese al punto inicial.

\includegraphics[width=.5\textwidth]{img/Francia} \centering
\end{frame}

\begin{frame}{Algoritmos}
\begin{description}
 \item[Entrada:] Ficheros con ciudades indicadas como puntos en el plano según sus
 coordenadas.
 \item[Salida:] \texttt{vector<int>} con el orden en el que se recorren las ciudades.
\end{description}
\end{frame}

% TODO: todo lo posterior

\begin{frame}{\texttt{Grafo}}
Un grafo consta de:

\begin{itemize}
  \item Una \textbf{cantidad de nodos}: \texttt{nodos}.
  \item Una \textbf{matriz de pesos}: \texttt{lados}.
\end{itemize}
\end{frame}

\subsection{Ramificación y acotación}

% TODO: Diapositiva previa con la idea del algoritmo?

\begin{frame}[fragile]{Cola con prioridad}
  Necesitamos una cola ordenada por la cota. \texttt{priority\_queue} no permite borrar nodos: implementamos nuestra clase con ese método:

  \lstinputlisting[firstline=31, lastline=33]{cpps/cola.h}

  Un funtor (\texttt{comp}) implementa la comparación. Guardamos en la cola el valor de la cota de cada nodo.

  \note{El uso de memoria era excesivo y hacía que las inserciones y borrados fueran demasiado lentos. Esta nueva operación mejora en \texttt{ulysses16} de más de 15 minutos a menos de un minuto.}
\end{frame}

\begin{frame}{Algoritmo general}
  Iniciamos:
  \begin{itemize}
    \item La cola con la solución parcial \texttt{[0]}
    \item La mejor solución con un algoritmo \textit{greedy} (vecino más cercano)
  \end{itemize}
\end{frame}

\begin{frame}{Algoritmo general}
  Mientras la cola no esté vacía, coge el primer elemento:
  \begin{itemize}
    \item Si \textbf{puede formarse una solución completa}, comprueba si es mejor que la mejor encontrada. En tal caso actualizala y borra los nodos con peor cota.
    \item En \textbf{otro caso}, para cada ciudad no visitada forma una nueva solución parcial. Añádela a la cola si su cota es mejor que la mejor solución.
  \end{itemize}
\end{frame}

\subsubsection{Cotas}

\begin{frame}{Cota del mínimo}
  Calcular la longitud del recorrido inicial y suma la menor distancia de cada ciudad no incluida en la solución parcial con sus relacionadas.
\end{frame}

\begin{frame}{Arbol generador}
  % TODO (Antonio): Cota del árbol generador minimal
\end{frame}

\subsection{Vuelta atrás}

%% TODO: (Iñaki) Descripción del algoritmo de backtracking

\subsection{Comparativa de los algoritmos}

%%% TODO (Iñaki)
% Gráficas que comparen entre los 3 algoritmos:
% - Número de nodos expandidos
% - Tamaño máximo de la cola con prioridad
% - Número de veces que se realiza la poda
% - Tiempo empleado en resolver el problema
%%%
