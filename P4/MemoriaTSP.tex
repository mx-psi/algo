\section{El problema del viajante de comercio}

El problema del viajante de comercio consiste en hallar el recorrido con distancia
mínima en un conjunto de ciudades que pase por todas las ciudades y regrese al punto
inicial. \\

La implementación de los algoritmos es de la forma:
\begin{description}
 \item[Entrada:] Ficheros con ciudades indicadas como puntos en el plano según sus
 coordenadas.
 \item[Salida:] \texttt{vector<int>} con el orden en el que se recorren las ciudades.
\end{description}

En la salida nos ahorraremos repetir el primer nodo al final del recorrido.

Para todos los algoritmos utilizaremos una estructura de datos que nos permita manejar
el problema: la clase \texttt{Grafo} (en el fichero \texttt{grafo.h}). Un grafo consta
de:

\begin{itemize}
  \item Una \textbf{cantidad de nodos}, almacenada en el atributo \texttt{nodos}.
  \item Una \textbf{matriz de pesos}, almacenada en el vector \texttt{lados}.
\end{itemize}

La interfaz nos permite acceder y modificar estos datos con mayor facilidad.
Mediante los métodos \texttt{setPeso} y \texttt{peso} accederemos al peso de un
lado del grafo, y el método \texttt{pesosDesdeCoordenadas} nos permite inicializar el
grafo utilizando el formato de datos en el que aparece el problema: calculando
la distancia entre cualesquiera dos ciudades y añadir esta como peso de ese lado:

\lstinputlisting[firstline=47, lastline=51]{cpps/grafo.h}

Adicionalmente, la función \texttt{longitud} calcula la longitud de un camino dado,
teniendo en cuenta la distancia del último nodo al primero:

\lstinputlisting[firstline=54, lastline=59]{cpps/grafo.h}

Siendo \texttt{peso\_t} el tipo de dato con el que se manejen las distancias entre
nodos. Según el enunciado del problema (que pide que se redondee la distancia
euclídea al entero más próximo), será de tipo \texttt{int}.

% TODO todo lo posterior.
% Además habría que revisar todo lo de arriba (está igual que en la práctica 3)