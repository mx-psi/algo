\section{El elemento en su posición}

La implementación de todos los algoritmos realizados es de la forma:
\begin{description}
 \item[Entrada:] Vector \texttt{v} y su tamaño \texttt{n}
 \item[Salida:] Entero no negativo que indica el $i$ tal que $v[i]=i$ en caso de que exista o $-1$ en oro caso
\end{description}

\subsection{Algoritmo obvio}

El algoritmo obvio que resuelve el problema de \textit{El elemento en su posición} consiste
en recorrer cada elemento del vector y comprobar para cada uno de estos si se cumple la
condición deseada ($v[i] = i$):

% Versión obvia
\lstinputlisting[firstline=23, lastline=28]{cpps/posicion.cpp}

Las condiciones de comienzo, actualización y final del bucle son todas $O(1)$, así como el código del bucle. El bucle se ejecutará un máximo de \texttt{n} veces, por lo que es claro que la eficiencia de este algoritmo es de $\mathbf{O(n)}$.

Los datos de la eficiencia empírica en este algoritmo queda:

\pgfplotstabletypeset[
display columns/0/.style={column name=Tamaño},
display columns/1/.style={column name=Tiempo},
]{dats/posicion_t_50.dat}


% Versión recursiva
\lstinputlisting[firstline=39, lastline=62]{cpps/posicion.cpp}
