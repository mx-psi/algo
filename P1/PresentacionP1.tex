\documentclass[10pt,compress,usetitleprogressbar,mathserif]{beamer}
\usepackage[spanish, es-tabla,es-noquoting,es-noshorthands]{babel}
\usepackage{tikz}
\usepackage{showexpl}
\usepackage{amsthm}
\usepackage{amsmath}
\usepackage{amssymb}

% Solarized palette
\definecolor{solarizedBase03}{HTML}{002B36}
\definecolor{solarizedBase02}{HTML}{073642}
\definecolor{solarizedBase01}{HTML}{586e75}
\definecolor{solarizedBase00}{HTML}{657b83}
\definecolor{solarizedBase0}{HTML}{839496}
\definecolor{solarizedBase1}{HTML}{93a1a1}
\definecolor{solarizedBase2}{HTML}{EEE8D5}
\definecolor{solarizedBase3}{HTML}{FDF6E3}
\definecolor{solarizedYellow}{HTML}{B58900}
\definecolor{solarizedOrange}{HTML}{CB4B16}
\definecolor{solarizedRed}{HTML}{DC322F}
\definecolor{solarizedMagenta}{HTML}{D33682}
\definecolor{solarizedViolet}{HTML}{6C71C4}
\definecolor{solarizedBlue}{HTML}{268BD2}
\definecolor{solarizedCyan}{HTML}{2AA198}
\definecolor{solarizedGreen}{HTML}{859900}

\setbeamercovered{dynamic}

\usetheme{epstfg}
\setbeamertemplate{note page}[compress]
\title{Práctica 1}
\author{Pablo Baeyens \and Antonio Checa \and Iñaki Madinabeitia \and Jose Manuel Muñoz \and Dario Sierra}
\date{Algorítmica}
\def\inline{\lstinline[basicstyle=\ttfamily]}

\begin{document}
\maketitle
\section{Ejercicio 1: \large{Eficiencia Empírica }}
\begin{frame}{Algoritmos Cuadráticos}
\begin{columns}[c]
	\column{.2\textwidth}
		\begin{itemize}
			\item	\only<1>{\large{Selección}}\only<2>{\large{Burbuja}}\only<3>{\large{Inserción}}
		\end{itemize}
		
	\column{.2\textwidth}
	Tablas
\end{columns}	
\end{frame}

\begin{frame}{Algoritmos Cúbicos}
	\begin{columns}[c]
		\column[]{.2\textwidth}
		\begin{itemize}
			\item \Large{Floyd}
		\end{itemize}
		\column[]{.3\textwidth}
		Tablas
	\end{columns}
\end{frame}


\begin{frame}{Algoritmos $n\cdot log(n)$}
	\begin{columns}[c]
		\column{.2\textwidth}

	\only<1>\Large{Mergesort}
		
		\column{.2\textwidth}
		Tablas
	\end{columns}	
\end{frame}

\begin{frame}{Fibonacci $t_n=t_{n-1}+t_{n-2}$}
	contenidos...
\end{frame}

\begin{frame}{Hanói $t_n=2 \cdot t_{n-1}+1 $}
	contenidos...
\end{frame}

\section{Ejercicio 2: Elaboración de Gráficas}

\begin{frame}{\small{Comparativa: Algoritmos Cuadráticos}}
	contenidos...
\end{frame}

\begin{frame}{Algoritmo de Floyd}
	contenidos...
\end{frame}

\begin{frame}{Fibonacci}
	contenidos...
\end{frame}

\begin{frame}{Hanói}
	contenidos...
\end{frame}

\begin{frame}{\small{Comparativa global de los algoritmos}}
	contenidos...
\end{frame}


\end{document}
